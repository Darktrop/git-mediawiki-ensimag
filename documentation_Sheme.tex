\documentclass[11pt]{article}

\usepackage[french]{babel}
\usepackage[utf8]{inputenc}
\usepackage{fancyhdr}
\usepackage{fourier}
\usepackage{lastpage}
\usepackage{placeins}
\usepackage{subfigure}
\usepackage[pdftex]{graphicx}
\usepackage{float}
% des commandes utiles pour ecrire des maths
\newcommand{\dx}{\,dx}
\newcommand{\dt}{\,dt}
\newcommand{\ito}{,\dotsc,}
\newcommand{\R}{\mathds{R}}
\newcommand{\N}{\mathds{N}}
\newcommand{\Poly}[1]{\mathcal{P}_{#1}}
\newcommand{\abs}[1]{\left\lvert#1\right\rvert}
\newcommand{\norm}[1]{\left\lVert#1\right\rVert}
\newcommand{\pars}[1]{\left(#1\right)}
\newcommand{\bigpars}[1]{\bigl(#1\bigr)}
\newcommand{\set}[1]{\left\{#1\right\}}
\newenvironment{Reponse}[1]{
\par
\setlength\fboxrule{2pt}%
\begin{minipage}{.95\linewidth}
\fbox{\parbox[c]{.972\linewidth}{\textbf{Propriété} \\ \medskip  #1}} 
\end{minipage}
}


\newenvironment{Reponse2}[1]{%
    %\vspace{3pt}
    \par
  \begin{minipage}{.95\linewidth}
    %\vspace{12pt}\par
    \xdef\@formanswerline{\@questionheader}%
    \xdef\@maskanswerline{\@questionheader}%
    \fbox{\parbox[c]{.972\linewidth}{\textbf{Q\thequestion} : #1}}
    \vspace\questionspace\par
    %{\@answerstitlefont\@answerstitle}
    \vspace{-2ex}
    \begin{multicols}{5}
    \begin{list}{\@answernumberfont\Alph{@choice}.~}
    	{\addtolength{\leftmargin}{-3pt}%
    	%\setlength{\labelsep}{-1pt}
    	\usecounter{@choice}}}{%
    \end{list}
    \end{multicols}
    \@initorcheck%
    \addtocontents{frm}{\@formanswerline\protect\\\protect\hline}%
    \addtocontents{msk}{\@maskanswerline\protect\\\protect\hline}%
  \end{minipage}
  }
  

%%%%%%
% Pour mise-en-forme des fichiers Ada
%
% voir exemple en fin de ce fichier.
%
% ATTENTION, requiert encoding utf-8 (voir 2ième "\lstset" ci-dessous)
 
\usepackage{listings}
\lstset{
  morekeywords={abort,abs,accept,access,all,and,array,at,begin,body,
      case,constant,declare,delay,delta,digits,do,else,elsif,end,entry,
      exception,exit,for,function,generic,goto,if,in,is,limited,loop,
      mod,new,not,null,of,or,others,out,package,pragma,private,
      procedure,raise,range,record,rem,renames,return,reverse,select,
      separate,subtype,task,terminate,then,type,use,when,while,with,
      xor,abstract,aliased,protected,requeue,tagged,until,public,static,void},
  sensitive=f,
  morecomment=[l]\#,
  morestring=[d]",
  showstringspaces=false,
  basicstyle=\small\ttfamily,
  keywordstyle=\bf\small,
  commentstyle=\itshape,
  stringstyle=\sf,
  extendedchars=true,
  columns=[c]fixed,
  numbers = left,
  numberstyle=\tiny,
  stepnumber=2,
  numbersep=5pt,
  framexleftmargin=5mm,
  language=JAVA,
  frame=shadowbox
}

% CI-DESSOUS: conversion des caractères accentués UTF-8 
% en caractères TeX dans les listings...
\lstset{
  literate=%
  {À}{{\`A}}1 {Â}{{\^A}}1 {Ç}{{\c{C}}}1%
  {à}{{\`a}}1 {â}{{\^a}}1 {ç}{{\c{c}}}1%
  {É}{{\'E}}1 {È}{{\`E}}1 {Ê}{{\^E}}1 {Ë}{{\"E}}1% 
  {é}{{\'e}}1 {è}{{\`e}}1 {ê}{{\^e}}1 {ë}{{\"e}}1%
  {Ï}{{\"I}}1 {Î}{{\^I}}1 {Ô}{{\^O}}1%
  {ï}{{\"i}}1 {î}{{\^i}}1 {ô}{{\^o}}1%
  {Ù}{{\`U}}1 {Û}{{\^U}}1 {Ü}{{\"U}}1%
  {ù}{{\`u}}1 {û}{{\^u}}1 {ü}{{\"u}}1%
}

%%%%%%%%%%
% TAILLE DES PAGES (A4 serré)

\setlength{\parindent}{0pt}
\setlength{\parskip}{1ex}
\setlength{\textwidth}{17cm}
\setlength{\textheight}{23cm}
\setlength{\oddsidemargin}{-.7cm}
\setlength{\evensidemargin}{-.7cm}
\setlength{\topmargin}{-.5in}

%%%%%%%%%%
% EN-TÊTES ET PIED DE PAGES

\pagestyle{fancyplain}
\renewcommand{\headrulewidth}{0pt}
\addtolength{\headheight}{1.6pt}
\addtolength{\headheight}{2.6pt}
\lfoot{}
\cfoot{}
\rfoot{\footnotesize\sf page~\thepage/\pageref{LastPage}}
\lhead{\footnotesize\sf Projet de spécialité}
\rhead{\footnotesize\sf ENSIMAG-2A} % numéro d'équipe Teide 

% titre, auteur et date
\title{Organisation pour le projet de spécialité\\
Passerelle Git-MediaWiki\\
~\\
\includegraphics[scale=0.75]{logo_ensimag.jpg} 
}

%% penser à indiquer: numéro d'équipe Teide + noms + groupe TD
\author{Julien Khayat, Guillaume Sasdy, Simon Perrat, Charles Roussel, Simon Cathébras}
\date{Le \today}
\begin{document}
\maketitle

\section{Description du projet et contexte}
\subsection{La bonne idée de \textit{Git-MediaWiki}}
Habituellement les pages web d'un wiki se modifient directement dans le navigateur, comme pour \textit{Wikipédia}. L'année dernière une équipe d'élèves de l'école a développé \textit{Git-Mediawiki} à l'initiative de Matthieu Moy ; cet outil permet de modifier les pages d'un wiki sous forme de fichiers et de répertoires gérés par le logiciel \textit{open source} de gestion de version \textit{Git}. Ainsi nous profitons des (divines) fonctionnalités d'un tel gestionnaire comme la possibilité d'avoir plusieurs versions d'un wiki à un instant donné.\\
\subsection{La validation d'un projet dans la communauté \textit{Git}}
Avant de voir \textit{Git-Mediawiki} ajouté à la version officielle de \textit{Git}, la communauté doit s'assurer de la qualité du code développé et de l'absence de bug. Il y a plusieurs étapes : tout d'abord, une proposition de patch est envoyée sur la mailing-liste de \textit{Git} pour recueillir les avis des contributeurs ; une fois que le patch a atteint une forme convenable, le code est ajouté à la branche \textit{pu} (pour \textit{proposed updates}), où il peut être retravaillé ; une fois les critiques prises en compte et le code corrigé, celui-ci est ajouté à la branche \textit{next} ; à partir de là, c'est en bonne voie pour que le mainteneur décide de l'ajouter à la branche \textit{master} (version officielle de \textit{Git}), souvent plusieurs mois après, le temps que le code "mûrisse". Aujourd'hui, le code de \textit{Git-Mediawiki} est dans le répertoire contrib/ de la branche \textit{master}.\\
\subsection{Notre projet : développer l'environnement de test}
Cette année, notre équipe développe un environnement de test pour \textit{Git-Mediawiki}. Il y a deux raisons à cela.
\begin{enumerate}
\item Lors de tout développement informatique sérieux, tester le logiciel est au moins aussi long que le développer. Cela permet de s'assurer qu'on ne remet pas "une bombe" entre les mains de l'utilisateur. Notre base de test permettra à la communauté \textit{Git} de s'assurer que \textit{Git-MediaWiki} est un logiciel fiable ou, cas échéant, savoir le corriger pour qu'il le devienne. À terme, il serait bon que celui-ci soit intégré à la version officielle de \textit{Git}, et il est indispensable pour cela d'avoir une solide base de tests.
\item Les développeurs qui se chargent d'ajouter des fonctionnalités pourront développer en \textit{Test Driven Development}, une méthode agile qui produit un code plus fiable et plus rapidement qu'avec les méthodes de développements classiques.
\end{enumerate}
Notre travail implique donc des interactions avec la communauté mais aussi avec l'autre équipe qui travaille sur \textit{Git-Mediawiki}, chargée d'ajouter des fonctionnalités à l'existant. Toutes deux, nous utilisons les méthodes agiles. Cela sera présenté en détails dans une autre section de ce document.

\section{Objectifs et périmètre}
\subsection{Objectifs}
Nous avons découpé notre projet en trois objectifs réalisables et mesurables grâce à aux critères de performance que nous détaillons également.
\begin{enumerate}
\item \textbf{Acceptation de notre environnement de tests par la communauté \textit{Git}}. Nous souhaitons que notre base de test soit sur la branche \textit{next} (cf. ce qui est expliqué dans le contexte du projet). Un développement accepté sur la branche \textit{pu}, sera considéré comme un objectif atteint à 50\%
\item \textbf{Couverture complète du code de base de \textit{Git-MediaWiki}}. Le critère de performance de ce point est simplement le nombre de fonctionnalités qui sont testées et approuvées de façon satisfaisante.
\item \textbf{Conformité de nos tests avec les standards des tests \textit{Git}}. En effet, la communauté Git a adopté des standards pour l'écriture des tests du logiciel. Il nous faudra nous y conformer. La mesure de ce critère est moins objective, nous considérerons les retours de la communauté par rapport à la conformité.
\end{enumerate}
Dans tous les cas, la meilleur référence pour juger de l'avancement de nos objectifs sera l'ensemble des avis de la communauté qui développe \textit{Git} car ils sont des développeurs expérimentés.\\

\subsection{Périmètre}
\subsubsection{L'environnement}
L'environnement de tests que nous allons mettre en place consiste essentiellement en des fonctions de manipulation de wiki. Nous avons également mis en place des fonctions de tests de contenu et de présence de fichiers. Ces fonctions sont utilisées pour faciliter la vérification de la cohérence entre un wiki de test et une archive \textit{Git}.
\begin{enumerate}
\item Ajout/édition d'une page
\item Suppression d'une page
\item Récupération d'une page
\item Comparaison de deux fichiers
\item Recherche d'un fichier par son nom
\end{enumerate}
\subsubsection{Les tests}
Les tests que nous allons écrire serviront dans un premier temps à valider le code déjà écrit, c'est à dire les fonctionnalités basiques de la passerelle comme le \textit{clone} (transformer un wiki en un dossier \textit{Git}), le \textit{push} (transformer les fichiers du dossier \textit{Git} comme pages du wiki), et le \textit{pull} (transformer les pages du wiki en fichiers \textit{Git}).

Une fois que ces fonctionnalités seront validées, nous planifions d'écrire des tests pour l'autre équipe de développement de la passerelle. Ce point est à la frontière du périmètre, c'est à dire que nous le réaliserons seulement si nos objectifs sont terminés dans les délais. Toutefois, si nous arrivons à ce stade, nous envisageons de corriger les bugs du code existant que nous parviendrons à détecter avec l'environnement de test. Il nous sera également possible de prêter main forte à l'équipe qui développe de nouvelles fonctionnalités pour \textit{Git-MediaWiki}.

\section{Organisation de l'équipe}
\subsection{Dates clef}
Notre projet se déroule avec les principes des méthodes agiles, les dates importantes prévues sont les suivantes :

\begin{itemize}
\item Réunion de fin de sprint du Vendredi 25 Mai
\item Réunion de fin de sprint du Lundi 4 Juin
\item Réunion de fin de sprint du Lundi 11 Juin
\item Rendu du rapport écrit le Jeudi 14 Juin
\item Soutenance le Vendredi 15 Juin
\end{itemize}

Pour s'assurer que nous respectons ce planning du mieux possible, nous avons mis en œuvre plusieurs moyen agiles.\\
\begin{itemize}
\item Burndown
\item Tableau de post-its de la méthode Scrum
\end{itemize}

\subsection{Organisation au quotidien}
\paragraph{Stand-up Meeting}
Chaque matin, l'équipe se réunit à heure fixe pour Stand-up Meeting.
Chacun choisit les tâches qu'il remplira dans la journée, et signale à l'équipe si il rencontre des difficultés. 
\paragraph{Réunion de fin de Sprint}
Cette réunion se fait une fois par sprint avec le product-owner. Cette réunion a pour but de réaliser une démonstration de ce qui a été réalisé dans le sprint écoulé.
\paragraph{Interactions}
Notre projet dépend de deux facteurs principaux. Notre product-owner, Matthieu Moy, d'une part, et d'autre part la communauté de développement de Git.\\
Un objectif important dans notre projet est de réussir à interagir au mieux avec la communauté, essentiellement par mail : c'est en effet nécessaire si l'on veut voir notre projet retenu pour la prochaine release de Git.
\subsection{Risques pour le projet}
Voici une liste des principaux risques que nous avons identifiés pour notre projet, ainsi que des exemples de solution.
\begin{enumerate}
\item La communauté n'est tout simplement pas intéressée par notre projet. 
\item Nos travaux sont envoyés trop tard pour bénéficier de retours.
\item L'équipe fonctionnalité est mise en retard pour réaliser des tests.
\end{enumerate}

Solutions possibles :
\begin{enumerate}
\item Envoyer un mail au plus tôt avec des exemples sur nos travaux. Être actifs pour promouvoir notre projet.
\item Faire de l'envoi à la Mailing-List une priorité.
\item Choix d'un représentant auprès de l'équipe fonctionnalités. Ce poste est assuré par Charles Roussel. 
\end{enumerate}

\section{Conclusion}
Pour beaucoup d'entre nous, ce sera notre premier projet avec les méthodes agiles. Une première découverte donc. De plus, nous allons tous, pour la première fois, développer pour un logiciel \textit{open source}. C'est-à-dire soumettre notre code à des développeurs expérimentés et utiliser une mailing-list, mais aussi se conformer à des standards existants. L'autre grande valeur ajoutée de ce projet de spécialité est que, pour la première fois, nous allons être en plein dans la validation de logiciel existant. Une activité non négligeable lors du développement que, jusque là, nous n'avons que très peu expérimentée lors de nos projets scolaires à l'Ensimag. 

\end{document}

\documentclass[11pt]{article}

\usepackage[french]{babel}
\usepackage[utf8]{inputenc}
\usepackage{fancyhdr}
\usepackage{fourier}
\usepackage{lastpage}
\usepackage{placeins}
\usepackage{subfigure}
\usepackage[pdftex]{graphicx}
\usepackage{float}
% des commandes utiles pour ecrire des maths
\newcommand{\dx}{\,dx}
\newcommand{\dt}{\,dt}
\newcommand{\ito}{,\dotsc,}
\newcommand{\R}{\mathds{R}}
\newcommand{\N}{\mathds{N}}
\newcommand{\Poly}[1]{\mathcal{P}_{#1}}
\newcommand{\abs}[1]{\left\lvert#1\right\rvert}
\newcommand{\norm}[1]{\left\lVert#1\right\rVert}
\newcommand{\pars}[1]{\left(#1\right)}
\newcommand{\bigpars}[1]{\bigl(#1\bigr)}
\newcommand{\set}[1]{\left\{#1\right\}}
\newenvironment{Reponse}[1]{
\par
\setlength\fboxrule{2pt}%
\begin{minipage}{.95\linewidth}
\fbox{\parbox[c]{.972\linewidth}{\textbf{Propriété} \\ \medskip  #1}} 
\end{minipage}
}


\newenvironment{Reponse2}[1]{%
    %\vspace{3pt}
    \par
  \begin{minipage}{.95\linewidth}
    %\vspace{12pt}\par
    \xdef\@formanswerline{\@questionheader}%
    \xdef\@maskanswerline{\@questionheader}%
    \fbox{\parbox[c]{.972\linewidth}{\textbf{Q\thequestion} : #1}}
    \vspace\questionspace\par
    %{\@answerstitlefont\@answerstitle}
    \vspace{-2ex}
    \begin{multicols}{5}
    \begin{list}{\@answernumberfont\Alph{@choice}.~}
    	{\addtolength{\leftmargin}{-3pt}%
    	%\setlength{\labelsep}{-1pt}
    	\usecounter{@choice}}}{%
    \end{list}
    \end{multicols}
    \@initorcheck%
    \addtocontents{frm}{\@formanswerline\protect\\\protect\hline}%
    \addtocontents{msk}{\@maskanswerline\protect\\\protect\hline}%
  \end{minipage}
  }
  

%%%%%%
% Pour mise-en-forme des fichiers Ada
%
% voir exemple en fin de ce fichier.
%
% ATTENTION, requiert encoding utf-8 (voir 2ième "\lstset" ci-dessous)
 
\usepackage{listings}
\lstset{
  morekeywords={abort,abs,accept,access,all,and,array,at,begin,body,
      case,constant,declare,delay,delta,digits,do,else,elsif,end,entry,
      exception,exit,for,function,generic,goto,if,in,is,limited,loop,
      mod,new,not,null,of,or,others,out,package,pragma,private,
      procedure,raise,range,record,rem,renames,return,reverse,select,
      separate,subtype,task,terminate,then,type,use,when,while,with,
      xor,abstract,aliased,protected,requeue,tagged,until,public,static,void},
  sensitive=f,
  morecomment=[l]\#,
  morestring=[d]",
  showstringspaces=false,
  basicstyle=\small\ttfamily,
  keywordstyle=\bf\small,
  commentstyle=\itshape,
  stringstyle=\sf,
  extendedchars=true,
  columns=[c]fixed,
  numbers = left,
  numberstyle=\tiny,
  stepnumber=2,
  numbersep=5pt,
  framexleftmargin=5mm,
  language=JAVA,
  frame=shadowbox
}

% CI-DESSOUS: conversion des caractères accentués UTF-8 
% en caractères TeX dans les listings...
\lstset{
  literate=%
  {À}{{\`A}}1 {Â}{{\^A}}1 {Ç}{{\c{C}}}1%
  {à}{{\`a}}1 {â}{{\^a}}1 {ç}{{\c{c}}}1%
  {É}{{\'E}}1 {È}{{\`E}}1 {Ê}{{\^E}}1 {Ë}{{\"E}}1% 
  {é}{{\'e}}1 {è}{{\`e}}1 {ê}{{\^e}}1 {ë}{{\"e}}1%
  {Ï}{{\"I}}1 {Î}{{\^I}}1 {Ô}{{\^O}}1%
  {ï}{{\"i}}1 {î}{{\^i}}1 {ô}{{\^o}}1%
  {Ù}{{\`U}}1 {Û}{{\^U}}1 {Ü}{{\"U}}1%
  {ù}{{\`u}}1 {û}{{\^u}}1 {ü}{{\"u}}1%
}

%%%%%%%%%%
% TAILLE DES PAGES (A4 serré)

\setlength{\parindent}{0pt}
\setlength{\parskip}{1ex}
\setlength{\textwidth}{17cm}
\setlength{\textheight}{23cm}
\setlength{\oddsidemargin}{-.7cm}
\setlength{\evensidemargin}{-.7cm}
\setlength{\topmargin}{-.5in}

%%%%%%%%%%
% EN-TÊTES ET PIED DE PAGES

\pagestyle{fancyplain}
\renewcommand{\headrulewidth}{0pt}
\addtolength{\headheight}{1.6pt}
\addtolength{\headheight}{2.6pt}
\lfoot{}
\cfoot{}
\rfoot{\footnotesize\sf page~\thepage/\pageref{LastPage}}
\lhead{\footnotesize\sf Projet GL}
\rhead{\footnotesize\sf ENSIMAG-2A} % numéro d'équipe Teide 

% titre, auteur et date
\title{Documentation technique du compilateur DecaC\\
~\\
\includegraphics[scale=0.75]{logo_ensimag.jpg} 
}

%% penser à indiquer: numéro d'équipe Teide + noms + groupe TD
\author{Julien Khayat, Guillaume Sasdy, Simon Perrat, Charles Roussel, Simon Cathébras}
\date{Le \today}
\begin{document}

Description du projet
Git. Wiki. Open source, logiciel libre. Tout le monde peut apporter sa contribution. Passerelle développé l'année dernière(Definition code de base):  éditer un wiki dans navigateur web -> image miroir sous forme de fichiers et répertoires afin de profiter des                fonctionnalités de Git (explicité). Avantages.
Notre équipe: environnement de test. Pourquoi ? Validation par la communauté du code de l'année dernière afin qu'il soit     intégré à la version "officielle" du logiciel Git.
Autre équipe en parallèle qui fonctionnalités (equipe fonctionnalité).
Methode agile. Interraction communauté. Progression des branches de git.
!!!!!! On chie agile !!!!

Objectifs et périmètre
3 objectifs:
*) Acceptation de notre environnement par la communauté. Critère : notre code sur 'branche next'
*) Couverture de test pour le code de l'année dernière. Critère : nombre de fonctionnalités testées et validées par la       communauté Git.
*) Mise en forme de nos tests et respect des normes de la communauté : nombre de d'insultes ...
+indicateurs de perf pour ces objectif

Périmètre. Concerne les manipulations usuelles sur un wiki: ajout, édition, suppression d'une page et vérification de la     cohérence avec le repository git équivalent. (environnement)
test automatique pour passerelle (tests)
Tester le code de base. Technos utilisées : Perl + Shell.
Perimètre étendu : debug du code de base + aider l'équipe fonctionnalité.

Organistation de l'équipe et Plannification.
	Dates = les sprints.
	Preciser qu'on est en agile.
	Interlocuteur = Charles (equipe fonctionnalité).
	Reunion + reporting (cf notes).
	Organisation Scrum : Tableau des User Stories+Avancement+burndown.
	Communication : Avec M.Moy + Avec Communauté git (feedback).
	
	Prevention de risques et dépendances:
		-Manque d'interet du projet par la communauté. S => envoyer vite et 		etre actif.
		-Envoyer les travaux trop tard. S => bien s'organiser : envoi à 			chaque fin de sprint.
		-L'autre équipe depend de nous. S => choix d'un interlocuteur, 				developper l'environnement rapidement
	
BLA BLA de la fin (pipoooooooooooo)

\maketitle

\section{Objectifs et périmètre}
\subsection{Objectifs}
Nous avons découpé notre projet en trois objectifs classé dans l'ordre suivant.
\begin{enumerate}
\item Acceptation de notre environnement de tests par la communauté git. Plus précisément, nous souhaitons que cette base de test soit sur la branche Next. Un développement accepté sur la branche PU, sera considéré comme un objectif atteint à 50\%
\item Fournir à la communauté une couverture de test complètement du code de base. Le principal critère de réussite est la couverture du code.
\item Nos tests doivent être conforme aux standards de la base de test de Git.
\end{enumerate}
Pour juger l'avancement de ces objectifs, nous prennons comme référence les avis de la communauté de développement de Git.\\
\subsection{Périmètre}
\subsubsection{l'environnement}
L'environnement de test que nous allons mettre en place consiste essentiellement en des fonctions de manipulation de wiki. Nous avons également mis en place des fonctions de tests de contenu et de présence de fichiers. Ces fonctions sont utilisées pour faciliter la vérification de la cohérence entre un wiki de test et une archive git.
\begin{enumerate}
\item Ajout/édition d'une page
\item Suppression d'une page
\item Récupération d'une page
\item Comparaison de deux fichiers
\item Recherche d'un fichier par son nom
\end{enumerate}
\subsubsection{Les tests}
Les tests que nous allons écrire serviront dans un premier temps à valider le code déjà écrit. C'est à dire les fonctionnalités basiques de la passerelle comme le clone, le push, et le pull.

Une fois que ces fonctionnalités seront validées, nous planifions d'écrire des tests pour l'autre équipe de développement de la passerelle. Ce point est à la frontière du périmètre, c'est à dire que nous le réaliserons seulement si le projet peut être terminé dans les délais.




\end{document}
\documentclass[11pt]{article}

\usepackage[french]{babel}
\usepackage[utf8]{inputenc}
\usepackage{fancyhdr}
\usepackage{fourier}
\usepackage{lastpage}
\usepackage{placeins}
\usepackage{subfigure}
\usepackage[pdftex]{graphicx}
\usepackage{float}
% des commandes utiles pour ecrire des maths
\newcommand{\dx}{\,dx}
\newcommand{\dt}{\,dt}
\newcommand{\ito}{,\dotsc,}
\newcommand{\R}{\mathds{R}}
\newcommand{\N}{\mathds{N}}
\newcommand{\Poly}[1]{\mathcal{P}_{#1}}
\newcommand{\abs}[1]{\left\lvert#1\right\rvert}
\newcommand{\norm}[1]{\left\lVert#1\right\rVert}
\newcommand{\pars}[1]{\left(#1\right)}
\newcommand{\bigpars}[1]{\bigl(#1\bigr)}
\newcommand{\set}[1]{\left\{#1\right\}}
\newenvironment{Reponse}[1]{
  \par
  \setlength\fboxrule{2pt}%
  \begin{minipage}{.95\linewidth}
    \fbox{\parbox[c]{.972\linewidth}{\textbf{Propriété} \\ \medskip  #1}} 
  \end{minipage}
}


\newenvironment{Reponse2}[1]{%
  %\vspace{3pt}
  \par
  \begin{minipage}{.95\linewidth}
    %\vspace{12pt}\par
    \xdef\@formanswerline{\@questionheader}%
    \xdef\@maskanswerline{\@questionheader}%
    \fbox{\parbox[c]{.972\linewidth}{\textbf{Q\thequestion} : #1}}
    \vspace\questionspace\par
    %{\@answerstitlefont\@answerstitle}
    \vspace{-2ex}
    \begin{multicols}{5}
      \begin{list}{\@answernumberfont\Alph{@choice}.~}
    	{\addtolength{\leftmargin}{-3pt}%
    	  %\setlength{\labelsep}{-1pt}
    	  \usecounter{@choice}}}{%
      \end{list}
    \end{multicols}
    \@initorcheck%
    \addtocontents{frm}{\@formanswerline\protect\\\protect\hline}%
    \addtocontents{msk}{\@maskanswerline\protect\\\protect\hline}%
  \end{minipage}
}


%%%%%%
% Pour mise-en-forme des fichiers Ada
%
% voir exemple en fin de ce fichier.
%
% ATTENTION, requiert encoding utf-8 (voir 2ième "\lstset" ci-dessous)

\usepackage{listings}
\lstset{
  morekeywords={abort,abs,accept,access,all,and,array,at,begin,body,
    case,constant,declare,delay,delta,digits,do,else,elsif,end,entry,
    exception,exit,for,function,generic,goto,if,in,is,limited,loop,
    mod,new,not,null,of,or,others,out,package,pragma,private,
    procedure,raise,range,record,rem,renames,return,reverse,select,
    separate,subtype,task,terminate,then,type,use,when,while,with,
    xor,abstract,aliased,protected,requeue,tagged,until,public,static,void},
  sensitive=f,
  morecomment=[l]\#,
  morestring=[d]",
  showstringspaces=false,
  basicstyle=\small\ttfamily,
  keywordstyle=\bf\small,
  commentstyle=\itshape,
  stringstyle=\sf,
  extendedchars=true,
  columns=[c]fixed,
  numbers = left,
  numberstyle=\tiny,
  stepnumber=2,
  numbersep=5pt,
  framexleftmargin=5mm,
  language=JAVA,
  frame=shadowbox
}

% CI-DESSOUS: conversion des caractères accentués UTF-8 
% en caractères TeX dans les listings...
\lstset{
  literate=%
  {À}{{\`A}}1 {Â}{{\^A}}1 {Ç}{{\c{C}}}1%
  {à}{{\`a}}1 {â}{{\^a}}1 {ç}{{\c{c}}}1%
  {É}{{\'E}}1 {È}{{\`E}}1 {Ê}{{\^E}}1 {Ë}{{\"E}}1% 
  {é}{{\'e}}1 {è}{{\`e}}1 {ê}{{\^e}}1 {ë}{{\"e}}1%
  {Ï}{{\"I}}1 {Î}{{\^I}}1 {Ô}{{\^O}}1%
  {ï}{{\"i}}1 {î}{{\^i}}1 {ô}{{\^o}}1%
  {Ù}{{\`U}}1 {Û}{{\^U}}1 {Ü}{{\"U}}1%
  {ù}{{\`u}}1 {û}{{\^u}}1 {ü}{{\"u}}1%
}

%%%%%%%%%%
% TAILLE DES PAGES (A4 serré)

\setlength{\parindent}{0pt}
\setlength{\parskip}{1ex}
\setlength{\textwidth}{17cm}
\setlength{\textheight}{23cm}
\setlength{\oddsidemargin}{-.7cm}
\setlength{\evensidemargin}{-.7cm}
\setlength{\topmargin}{-.5in}

%%%%%%%%%%
% EN-TÊTES ET PIED DE PAGES

\pagestyle{fancyplain}
\renewcommand{\headrulewidth}{0pt}
\addtolength{\headheight}{1.6pt}
\addtolength{\headheight}{2.6pt}
\lfoot{}
\cfoot{}
\rfoot{\footnotesize\sf page~\thepage/\pageref{LastPage}}
\lhead{\footnotesize\sf Projet de spécialité}
\rhead{\footnotesize\sf ENSIMAG-2A} % numéro d'équipe Teide 

% titre, auteur et date
\title{User manual\\
  Git-Remote-Mediawiki, Test environment \\
  ~\\
  \includegraphics[scale=0.75]{logo_ensimag.jpg} 
}

%% penser à indiquer: numéro d'équipe Teide + noms + groupe TD
\author{Simon Cathébras, Julien Khayat, Simon Perrat, Charles Roussel,
  Guillaume Sasdy}

\begin{document}
\maketitle
\newpage
\section*{Introduction}
This manual describes how to install the git-remote-mediawiki test
environment on a machine with git installed on it.
\part*{Prerequisite}
\section{Get the sources}
To be able to run this test environment you first need to get the
sources developped in our project, which you can download at
\lstinline!https://github.com/Fafinou/git/!. Then, get our five main
patches from branch master.
\section{Packages required}
In order to run this test environment correctly, you will need to
install the following packages.
\begin{itemize}
\item php5-cli
\item lighttpd
\item php5-cgi
\item php5
\item php5-curl
\item php5-sqlite
\end{itemize}
\part*{Run the test environment}
\section{Install a new wiki}
Once you have all the prerequisite, you need to install a MediaWiki on
your machine; if you already have one, we strongly recommend you to
install one with the script we provide. Here's how to work it:
\begin{itemize}
\item change directory to \lstinline!contrib/mw-to-git/t/!
\item edit \lstinline!test.config! to choose your installation parameters
\item run \lstinline!./install-wiki.sh install!
\item check on your favourite web browser if your wiki is correctly
  installed.
\end{itemize}

\section{Remove an existing wiki}
Simply edit the file \lstinline!test.config! to fit the wiki you want
to delete, and execute the command \lstinline!./install-wiki.sh
delete! into the \lstinline!contrib/mw-to-git/t! directory.
\section{Run the existing tests}
The furnished tests are currently in the
\lstinline!contrib/mw-to-git/t! directory. The files are all the
\lstinline!t936*! shell scripts.

\paragraph*{Run all tests}
To do so, simply go into the \lstinline!contrib/mw-to-git/! directory,
and do \lstinline!Make test!.

\paragraph*{Run a specific test}
To run a given test \lstinline!test_name!, go into the
\lstinline!contrib/mw-to-git/t! directory and run
\lstinline!./test_name!.

\part*{Create new tests}

\section{Available functions}
The test environment of git-remote-mediawiki provides you with some
functions useful to test the behaviour of git-remote-mediawiki. for
more details about the functions' parameters, please refer to the
\lstinline!test-gitmw.pl! file.

\paragraph*{\lstinline!test_check_wiki_precond!}
Checks if the tests must be skipped or not. Please use this function
at the beggining of each new test file.

\paragraph*{\lstinline!wiki_getpage!}
Fetchs a given page from the wiki and puts its content in the
directory in parameter.

\paragraph*{\lstinline!wiki_delete_page!}
Deletes a given page from the wiki.

\paragraph*{\lstinline!wiki_edit_page!}
Creates or modifies a given page in the wiki. You can specify several
parameters like a summary for the page edition, or add the page to a
given category.\\
Please, see \lstinline!test-gitmw.pl! for more details.

\paragraph*{\lstinline!wiki_getallpage!}
Fetchs all pages from the wiki into a given directory. The directory
is created if it does not exists.

\paragraph*{\lstinline!test_diff_directories!}
Compares the content of two directories. The content must be the same.
Use this function to compare the content of a git directory and a wiki
one created by \lstinline!wiki_getallpage!.

\paragraph*{\lstinline!test_contains_N_files!}
Checks if the given directory contains a given number of file.

\paragraph*{\lstinline!wiki_page_exists!}
Tests if a given page exists on the wiki.

\paragraph*{\lstinline!wiki_reset!}
Reset the wiki, ie flush the database. Use this function at the
begining of each new test.

\section{How to write a new test}
Please, follow the standards given by git. See \lstinline!git/t/README!.\\
Name your new file \lstinline!t936[...].sh!.
Be sure to reset your wiki regulary with the function \lstinline!wiki_reset!


\end{document}

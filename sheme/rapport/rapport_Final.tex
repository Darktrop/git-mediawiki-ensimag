\documentclass[11pt]{article}

\usepackage[french]{babel}
\usepackage[utf8]{inputenc}
\usepackage{fancyhdr}
\usepackage{fourier}
\usepackage{lastpage}
\usepackage{placeins}
\usepackage{subfigure}
\usepackage[pdftex]{graphicx}
\usepackage{float}
% des commandes utiles pour ecrire des maths
\newcommand{\dx}{\,dx}
\newcommand{\dt}{\,dt}
\newcommand{\ito}{,\dotsc,}
\newcommand{\R}{\mathds{R}}
\newcommand{\N}{\mathds{N}}
\newcommand{\Poly}[1]{\mathcal{P}_{#1}}
\newcommand{\abs}[1]{\left\lvert#1\right\rvert}
\newcommand{\norm}[1]{\left\lVert#1\right\rVert}
\newcommand{\pars}[1]{\left(#1\right)}
\newcommand{\bigpars}[1]{\bigl(#1\bigr)}
\newcommand{\set}[1]{\left\{#1\right\}}
\newenvironment{Reponse}[1]{
  \par
  \setlength\fboxrule{2pt}%
  \begin{minipage}{.95\linewidth}
    \fbox{\parbox[c]{.972\linewidth}{\textbf{Propriété} \\ \medskip  #1}} 
  \end{minipage}
}


\newenvironment{Reponse2}[1]{%
  %\vspace{3pt}
  \par
  \begin{minipage}{.95\linewidth}
    %\vspace{12pt}\par
    \xdef\@formanswerline{\@questionheader}%
    \xdef\@maskanswerline{\@questionheader}%
    \fbox{\parbox[c]{.972\linewidth}{\textbf{Q\thequestion} : #1}}
    \vspace\questionspace\par
    %{\@answerstitlefont\@answerstitle}
    \vspace{-2ex}
    \begin{multicols}{5}
      \begin{list}{\@answernumberfont\Alph{@choice}.~}
    	{\addtolength{\leftmargin}{-3pt}%
    	  %\setlength{\labelsep}{-1pt}
    	  \usecounter{@choice}}}{%
      \end{list}
    \end{multicols}
    \@initorcheck%
    \addtocontents{frm}{\@formanswerline\protect\\\protect\hline}%
    \addtocontents{msk}{\@maskanswerline\protect\\\protect\hline}%
  \end{minipage}
}


%%%%%%
% Pour mise-en-forme des fichiers Ada
%
% voir exemple en fin de ce fichier.
%
% ATTENTION, requiert encoding utf-8 (voir 2ième "\lstset" ci-dessous)

\usepackage{listings}
\lstset{
  morekeywords={abort,abs,accept,access,all,and,array,at,begin,body,
    case,constant,declare,delay,delta,digits,do,else,elsif,end,entry,
    exception,exit,for,function,generic,goto,if,in,is,limited,loop,
    mod,new,not,null,of,or,others,out,package,pragma,private,
    procedure,raise,range,record,rem,renames,return,reverse,select,
    separate,subtype,task,terminate,then,type,use,when,while,with,
    xor,abstract,aliased,protected,requeue,tagged,until,public,static,void},
  sensitive=f,
  morecomment=[l]\#,
  morestring=[d]",
  showstringspaces=false,
  basicstyle=\small\ttfamily,
  keywordstyle=\bf\small,
  commentstyle=\itshape,
  stringstyle=\sf,
  extendedchars=true,
  columns=[c]fixed,
  numbers = left,
  numberstyle=\tiny,
  stepnumber=2,
  numbersep=5pt,
  framexleftmargin=5mm,
  language=JAVA,
  frame=shadowbox
}

% CI-DESSOUS: conversion des caractères accentués UTF-8 
% en caractères TeX dans les listings...
\lstset{
  literate=%
  {À}{{\`A}}1 {Â}{{\^A}}1 {Ç}{{\c{C}}}1%
  {à}{{\`a}}1 {â}{{\^a}}1 {ç}{{\c{c}}}1%
  {É}{{\'E}}1 {È}{{\`E}}1 {Ê}{{\^E}}1 {Ë}{{\"E}}1% 
  {é}{{\'e}}1 {è}{{\`e}}1 {ê}{{\^e}}1 {ë}{{\"e}}1%
  {Ï}{{\"I}}1 {Î}{{\^I}}1 {Ô}{{\^O}}1%
  {ï}{{\"i}}1 {î}{{\^i}}1 {ô}{{\^o}}1%
  {Ù}{{\`U}}1 {Û}{{\^U}}1 {Ü}{{\"U}}1%
  {ù}{{\`u}}1 {û}{{\^u}}1 {ü}{{\"u}}1%
}

%%%%%%%%%%
% TAILLE DES PAGES (A4 serré)

\setlength{\parindent}{0pt}
\setlength{\parskip}{1ex}
\setlength{\textwidth}{17cm}
\setlength{\textheight}{23cm}
\setlength{\oddsidemargin}{-.7cm}
\setlength{\evensidemargin}{-.7cm}
\setlength{\topmargin}{-.5in}

%%%%%%%%%%
% EN-TÊTES ET PIED DE PAGES

\pagestyle{fancyplain}
\renewcommand{\headrulewidth}{0pt}
\addtolength{\headheight}{1.6pt}
\addtolength{\headheight}{2.6pt}
\lfoot{}
\cfoot{}
\rfoot{\footnotesize\sf page~\thepage/\pageref{LastPage}}
\lhead{\footnotesize\sf Projet de spécialité}
\rhead{\footnotesize\sf ENSIMAG-2A} % numéro d'équipe Teide 

% titre, auteur et date
\title{Rapport du projet de spécialité\\
  Passerelle Git-MediaWiki, environement de test\\
  ~\\
  \includegraphics[scale=0.75]{logo_ensimag.jpg} 
}

%% penser à indiquer: numéro d'équipe Teide + noms + groupe TD
\author{Simon Cathébras, Julien Khayat, Simon Perrat, Charles Roussel,
  Guillaume Sasdy}
\date{Le \today}
\begin{document}
\maketitle

\section{Travail réalisé}

Le travail réalisé durant ce projet est un environnement de test pour
la passerelle \textit{Git-MediaWiki}, un projet initié l'année
dernière à l'Ensimag.

\subsection{La passerelle  \textit{Git-MediaWiki}}

Cet outil a pour but de permettre de gérer un MediaWiki a l'aide de
Git. Pour bien comprendre de quoi il s'agit, nous allons expliquer
précisément ce que sont Git et MediaWiki

\subsubsection{MediaWiki : des sites communautaires à contribution libre}

Le terme wiki désigne un site web dont les pages peuvent être éditées
par les internautes qui le fréquentent, et MediaWiki est un type de
wiki particulier. Wikipédia, le wiki le plus fréquenté actuellement,
fonctionne avec MediaWiki, et on y retrouve toutes ses
caractéristiques, notamment~:


\begin{itemize}
\item chacun peut y contribuer de manière libre~;
\item possibilité d'ajouter, supprimer, et surtout éditer des pages~;
\item conservation de l'historique complet des modifications.
\end{itemize}

Le principal défaut de ces sites est que leur édition peut s'avérer
fastidieuse, et plusieurs problèmes se présentent régulièrement aux
gros utilisateurs~:

\begin{itemize}
\item conflits entre éditions concurrentes~;
\item la moindre petite modification de page, comme une correction
  orthographique, engendre une entrée dans l'historique~;
\item une seule révision de page entraîne une révision complète du
  wiki, l'historique est donc rapidement surchargé~;
\item ainsi pour une mise à jour importante (par exemple sur Wikipédia, un
  évènement d'actualité marquant), les pages touchées doivent être
  révisées une par une~;
\item de plus, dans le cas d'une modification conséquente sur une
  page, on est confronté à un dilemme~:
  \begin{itemize}
  \item soit on enregistre régulière et on produit un historique "sale"
  \item soit on n'enregistre qu'une fois le travail terminé, et on
    s'expose alors au risque de perdre son travail sur une coupure de
    connexion, un crash, une fermeture accidentelle du navigateur...
  \end{itemize}
\end{itemize}

Le but de la passerelle est alors de proposer une solution à ces
problèmes en permettant d'utiliser le gestionnaire de version
\textit{Git} pour travailler sur un MediaWiki.

\subsubsection{Git : un gestionnaire de version}

Un gestionnaire de version est un logiciel qui permet de partager et
synchroniser un ensemble de fichiers entre plusieurs
utilisateurs. Typiquement, un dépot est créé sur un serveur et toute
personne qui y a accès peut le copier sur son ordinateur personnel,
pour ensuite partager ses modifications sur les
fichiers. \textit{Git}, comme d'autres gestionnaires de version,
propose en outre les fonctionnalités suivantes~:

\begin{itemize}
\item possibilité d'avoir plusieurs versions d'un même fichier~;
\item gestion des éditions concurrentes~;
\item conservation d'un historique complet et détaillé.
\end{itemize}

On constate alors quelques avantages par rapport à un wiki~:

\begin{itemize}
\item on peut éditer un fichier en plusieurs fois en faisant une seule
  mise en ligne, l'historique résultant est alors plus propre~;
\item faire une mise à jour simultanée de plusieurs fichiers est plus simple~;
\item aucun risque de perte de donnée en cours d'édition, car la plupart
  des éditeurs de texte ont un système de sauvegarde automatique.
\end{itemize}

\subsection{L'environnement de tests}

L'outil \textit{Git-MediaWiki} est actuellement intégré à \textit{Git}
comme une contribution, sans être présent dans son coeur de
commandes. Pour pouvoir aller plus loin et intégrer nativement la
passerelle dans \textit{Git}, elle doit être munie d'une base de test
conséquente pour valider son fonctionnement. Jusqu'à présent, les
tests devaient être effectués à la main.

L'objectif de notre groupe est de construire un environnement complet
de tests pour la passerelle, avec pour objectif, à court terme de
valider les fonctionnalités existantes et d'en développer de
nouvelles, à long terme de pousser \textit{Git-MediaWiki} jusque
dans le coeur de \textit{Git}.

\subsubsection{Un wiki sur localhost}

Pour effectuer les tests de \textit{Git-MediaWiki}, un premier 
pré-requis est de posséder un wiki installé en local (par défaut). Pour éviter que les utilisateur aient à effectuer les pénibles étapes de création d'un wiki, l'équipe a décidé de fournir un script d'installation de wiki en local.

L'idée est de disposer ainsi d'un wiki connu avec lequel les tests 
peuvent interagir aisément. On fournit donc avec ce wiki un 
package complet de fonction pour manipuler ce wiki : ajout, 
supression et édition de page; récupération d'une page et de son 
contenu; comparaison d'une page et d'un fichier; etc...

Le wiki ainsi créé dispose en plus d'un fonction 
\lstinline!wiki_reset! qui permet d'assurer qu'un wiki est vierge
au début de chacun des tests composant la batterie de tests.

\subsubsection{Une Batterie de test}

Cette partie de notre travail est pour ainsi dire le coeur 
de notre projet. Nous proposons à l'utilisateur une batterie complète
de test pour \textit{Git-MediaWiki} activée via la commande 
\lstinline!Make test!. Ces tests sont conçu pour valider le 
fonctionnement de \textit{Git-MediaWiki}, avec 2 objectifs principaux.

\begin{enumerate}
\item Disposer d'une base de test de non régression pour les prochain 
  contributeurs à la passerelle. De plus, disposer d'un environnement de 
  tests complet permet d'utiliser facilement des méthodes agiles tel que
  le TDD (\textit{Test Driven Developpement}). \\
\item Détecter et référencer les bugs actuels de la passerelle à l'aide
  des fonctions de la base de test de qui, et ainsi posséder une base 
  solide du fonctionnement (ou non) de la passerelle selon les cas. De
  plus, si un développeur indépendant veut débugguer la passerelle, ces 
  tests lui seront très utile pour savoir si oui ou non ses modifications
  sont efficaces.
\end{enumerate}

\subsubsection{Une amélioration de performance}

En plus de l'environement de test, notre équipe a également développé 
une amélioration de performance pour \textit{Git-MediaWiki}. Selon le
type de wiki cloné, la passerelle actuelle peut parfois s'avérer très
lente pour des opérations simple (\textit{git pull}). L'idée de cette
optimisation est de déterminer l'évolution du wiki depuis la dernière
mise à jour du dépot Git, et de récupérer les modifications d'une 
manière différente selon celles ci. Actuellement, ce choix de méthode
doit être effectué par l'utilisateur, on ne dispose pas encore de
système automatisé de la solution à employer.  On distingue 2 cas :

\begin{itemize}
\item Dans le cas ou on a plus de révisions (i.e modification) du wiki 
  que de pages, on va procéder en récupérant la dernière mise à jour de chaque
  page. C'est la méthode présente nativement sur \textit{Git-MediaWiki}. Celle
  ci peut poser des problèmes dans le cas d'un wiki avec beaucoup de pages et
  peu actif.\\
\item L'autre possibilité est d'avoir moins de révision que de pages dans le
  wiki. Dans ce cas, la bonne méthode est non plus de regarder page par page,
  mais de consulter la liste des révisions et de ne récupérer que la plus récente
  pour chacune des page. Cette méthode est, au contraire de la précédente, très
  efficace pour des wiki peu actifs, mais va peiner sur des wiki plus ``gros". 
\end{itemize}

\section{La communautée \textit{Git}}

\textit{Git} est un logiciel libre et open source, ce qui signifie que
tout le monde a accès à son code source et peut y contribuer en
proposant des modifications. Il existe alors une communauté de
développeurs contribuant à \textit{Git}, qui communiquent ensemble via
une mailing-list dédiée et échangent ainsi sur les bugs rencontrés et
leur correction, les ajouts de fonctionnalités et les mises à jour du
code de manière générale. La version officielle est maintenue par
Junio~C.~Hamano, qui se charge de vérifier que les patches proposés
ont un niveau de finition suffisant pour être intégrés, en plus d'être
lui-même un des contributeurs principaux au code source de \textit{Git}.

\subsection{Organisation}

Le code de \textit{Git} est disponible librement sur une archive
\textit{Git}. Cette archive comprend trois branches, qui correspondent
chacune à différents avancements des patches proposés. La principale,
\textit{master}, est celle qui correspond à la version officielle
distribuée. La branche \textit{next} contient les patches qui seront
ajoutés à la prochaine version officielle. Enfin, la branche
\textit{pu}, pour \textit{proposed updates}, comprend les patches qui
sont en voie d'intégration à \textit{Git}, mais qui ont encore besoin
de peaufinage pour passer dans \textit{next}. Il faut noter qu'avant
même qu'un patch puisse passer dans cette branche, il doit déjà être
relu, testé et approuvé par la communauté sur la mailing-list.

\subsection{Interactions avec la communautée}

Nos échanges avec la communauté se font via la mailing-list de
\textit{Git}. Nos propositions de patches se font en plusieurs temps~:
tout d'abord, nous envoyons sur cette liste de diffusion une `cover
letter`, soit un mail où nous présentons notre équipe et notre
projet. Lorsque nos premiers travaux nous semblent assez mûrs, nous
envoyons alors un série de patches sur la mailing list, que la
communauté peut alors relire et commenter. Nous tenons alors compte
des critiques émises pour corriger notre code, et envoyons une
nouvelle série de patches. Nous continuons ainsi jusqu'à obtenir un
code qui atteigne les critères d'acceptation par \textit{Git}, et qui
soit en mesure d'être ajouté à la branche \textit{pu}.
Actuellement, nous n'avons pas atteint cette branche mais avons bon
espoir d'y parvenir.

\section{Difficultées pendant le projet}

\subsection{Premier Run : Lundi 21 Mai $\rightarrow$ Mardi 29 Mai}

Le premier run de ce projet de spécialité a comporté un nombre conséquent de problèmes, tant au niveau organisationnel que technique

\subsubsection{problèmes techniques}

Durant le premier run, nous avons eu des difficultés à appréhender le projet. Nous avons dû procéder à beaucoup de changements de structure pour notre code
\begin{enumerate}
\item plusieurs fichiers \textit{bash} encapsulant du \textit{Perl}
\item continuer les étapes
\end{enumerate}

\subsubsection{problèmes organisationnels}

Après une analyse à la fin du premier run, nous avons déterminé plusieurs problèmes d'organisation : 

\begin{itemize}
\item manque de communication dans le groupe
\item mauvaise compréhension de certains points de la charte d'équipe par une partie du groupe.
\item plusieurs membres du groupe ont du s'absenter un après midi ou une journée pour des raisons personnelles
\item pas d'évènement de cohésion de groupe
\end{itemize}

Tout ces points conduisant à un premiers run assez désorganisé, produisant des difficultés à apréhender le projet sur un plan général.

\begin{itemize}
\item stand-up meeting fixé à 9h35 chaque matin~;
\item petit-déjeuner d'équipe~;
\item utilisation systématique des post-it~;
\item pair programming et relecture de code obligatoire.
\end{itemize}

  Ces quelques règles nous ont permis de gagner significativement en
  efficacité lors du deuxième sprint.

\subsection{Deuxième sprint~: mardi 29 mai $\rightarrow$ mardi 5 juin}

Notre priorité lors de ce sprint était d'envoyer un premier patch sur
la mailing-list, afin d'avoir une idée de l'opinion que se fait la
communauté de notre projet. Cet objectif clair associé à notre
organisation nous a permi de travailler efficacement, et nous avons pu
envoyer la première version de notre batterie de tests sur la
mailing-list, puis prendre en compte les retours.

\subsection{Troisième sprint~: mardi 5 juin $\rightarrow$ mercredi 13 juin}

Ce sprint s'est nettement moins bien passé que le précédent~: malgré
un début de sprint où nous sommes restés sur la lancé du précédent et
avons continué à avancer, nous avons commencé à rencontrer des
problèmes techniques que nous n'avions pas sérieusement appréhendés au
début du projet, à savoir l'application de nos patches à d'autres
machines que les nôtres avec les contraintes techniques
associées. En plus de cela, nous avons eu des problèmes lorsqu'il a
fallu intégrer nos travaux respectifs ensemble, et nous avons perdu
énormément de temps à former une série de patch qui puisse être
envoyée sur la mailing-list. En fin de sprint, nous nous sommes donc
rendus compte d'une part que nous avions perdu certaines de nos bonnes
habitudes du sprint précédent, d'autre part qu'il nous a manqué
d'avoir certaines règles de codage qui auraient facilité l'intégration
de nos travaux.

\section{Bilan}

\subsection*{Au niveau de l'équipe}

\subsubsection*{Gestion d'équipe et méthodes agiles}

Tout au long du projet, la vie de l'équipe a été marqué par l'empreinte des
méthodes agiles. Ces techniques ont influencer notre manière d'apréhender 
le projet, de répartir les charges de travail, et de gérer l'équipe. Notre
gestion a souvent un avantage, mais parfois aussi un handicape. 

On retiendra que des règles de vie mal comprises par certains au début du 
projet ont ralentit le travail, alors qu'à la fin c'est la fatigue généralisé
qui a empêché un bon fonctionnement de la communication de groupe. Mais ces
problèmes ont gardé un impact réduit sur le fonctionnement de l'équipe. 

Au final, les bilans de fin de sprint on permit de mettre à plat les problèmes
et de leur trouver une solution convenant à l'ensemble du groupe.

On notera également que la pratique des méthodes agiles a été apréciée par 
l'ensemble du groupe, que ce soit ceux qui en avaient utilisé au projet GL,
ou ceux qui pour qui c'était nouveaux. le Scrumboard aura été un élément de 
référence tout au long du développement, servant de point de repère sur 
l'avancement des tâches.

\subsubsection*{Le logiciel libre }

Après une première approche du logiciel libre, l'équipe ressort de cette 
expérience avec un mélange de satisfaction et de frustration.

Satisfaction tout d'abord car ce premier contact nous a montré tous les 
avantages qu'apporte ce système, notamment grâce à la mailling-list qui 
permet de se lancer dans du code et garantit de bénéficier d'une revue de
qualité. Malgrès tout, on retiendra que l'accès à la mailling-list peut 
s'avérer compliquée au profane.

En effet, les contributeurs à \textit{Git} auront montré au cour de ce 
projet leur capacité a émettre des remarques constructives. Cette aide
a été inestimable tout au long du projet. L'équipe a réussi durant ces
trois semaine à s'intégrer au flux de mail, même si répondre sur d'autres
sujets que le notre aura été difficile.

Frustration devant un travail laissé partiellement inachevé, mais surtout
devant le peut d'intéré porté par la communauté à nos travaux. Le peu de
diversité parmis nos relecteurs nous a parfois poussé à nous demandé si
notre projet valait vraiment le coup. 

Au final cette expérience nous aura permit de découvrir à la fois les bon
et les mauvais côté du logiciel libre.


\subsection*{Au niveau des enseignants}

L'ensemble de l'équipe s'accorde à dire que l'enseignant de
  référence (Matthieu~Moy) a été un très bon encadrant tout au long du
  projet~: disponible, réactif sur les questions par mail et présent
  dans la salle. Il aura été d'une grande aide grace à sa connaissance
  des l'environnement du projet, c'est à dire la communauté Git et ses
  coutumes, le logiciel libre dans un sens plus général, mais
  également dans sa capacité à nous orienter dans la bonne direction
  en ce qui concerne l'architecture des fichiers.
  \\
  Un autre point très positif du projet aura été l'application des
  méthodes agiles, qui ont permi à l'équipe de travailler à la fois plus
  sereinement et plus efficacement. Les passages régulier de
  Jean-François~Jagodzinski nous ont incité à réfléchir à notre
  organisation, améliorant notre travail d'équipe tout au long du projet.

\subsection*{Améliorations}

Après discussion dans l'équipe voici quelques pistes d'amélioration pour le projet dans les années à venir : 

\begin{itemize}
\item donner des informations ``technique'' sur le projet sur la page de présentation. Il aurait put être interessant de connaitre les langages utilisés à l'avance. 
\end{itemize}

\end{document}

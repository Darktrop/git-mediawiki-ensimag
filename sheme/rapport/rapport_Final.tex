\documentclass[11pt]{article}

\usepackage[french]{babel}
\usepackage[utf8]{inputenc}
\usepackage{fancyhdr}
\usepackage{fourier}
\usepackage{lastpage}
\usepackage{placeins}
\usepackage{subfigure}
\usepackage[pdftex]{graphicx}
\usepackage{float}
% des commandes utiles pour ecrire des maths
\newcommand{\dx}{\,dx}
\newcommand{\dt}{\,dt}
\newcommand{\ito}{,\dotsc,}
\newcommand{\R}{\mathds{R}}
\newcommand{\N}{\mathds{N}}
\newcommand{\Poly}[1]{\mathcal{P}_{#1}}
\newcommand{\abs}[1]{\left\lvert#1\right\rvert}
\newcommand{\norm}[1]{\left\lVert#1\right\rVert}
\newcommand{\pars}[1]{\left(#1\right)}
\newcommand{\bigpars}[1]{\bigl(#1\bigr)}
\newcommand{\set}[1]{\left\{#1\right\}}
\newenvironment{Reponse}[1]{
  \par
  \setlength\fboxrule{2pt}%
  \begin{minipage}{.95\linewidth}
    \fbox{\parbox[c]{.972\linewidth}{\textbf{Propriété} \\ \medskip  #1}} 
  \end{minipage}
}


\newenvironment{Reponse2}[1]{%
  %\vspace{3pt}
  \par
  \begin{minipage}{.95\linewidth}
    %\vspace{12pt}\par
    \xdef\@formanswerline{\@questionheader}%
    \xdef\@maskanswerline{\@questionheader}%
    \fbox{\parbox[c]{.972\linewidth}{\textbf{Q\thequestion} : #1}}
    \vspace\questionspace\par
    %{\@answerstitlefont\@answerstitle}
    \vspace{-2ex}
    \begin{multicols}{5}
      \begin{list}{\@answernumberfont\Alph{@choice}.~}
    	{\addtolength{\leftmargin}{-3pt}%
    	  %\setlength{\labelsep}{-1pt}
    	  \usecounter{@choice}}}{%
      \end{list}
    \end{multicols}
    \@initorcheck%
    \addtocontents{frm}{\@formanswerline\protect\\\protect\hline}%
    \addtocontents{msk}{\@maskanswerline\protect\\\protect\hline}%
  \end{minipage}
}


%%%%%%
% Pour mise-en-forme des fichiers Ada
%
% voir exemple en fin de ce fichier.
%
% ATTENTION, requiert encoding utf-8 (voir 2ième "\lstset" ci-dessous)

\usepackage{listings}
\lstset{
  morekeywords={abort,abs,accept,access,all,and,array,at,begin,body,
    case,constant,declare,delay,delta,digits,do,else,elsif,end,entry,
    exception,exit,for,function,generic,goto,if,in,is,limited,loop,
    mod,new,not,null,of,or,others,out,package,pragma,private,
    procedure,raise,range,record,rem,renames,return,reverse,select,
    separate,subtype,task,terminate,then,type,use,when,while,with,
    xor,abstract,aliased,protected,requeue,tagged,until,public,static,void},
  sensitive=f,
  morecomment=[l]\#,
  morestring=[d]",
  showstringspaces=false,
  basicstyle=\small\ttfamily,
  keywordstyle=\bf\small,
  commentstyle=\itshape,
  stringstyle=\sf,
  extendedchars=true,
  columns=[c]fixed,
  numbers = left,
  numberstyle=\tiny,
  stepnumber=2,
  numbersep=5pt,
  framexleftmargin=5mm,
  language=JAVA,
  frame=shadowbox
}

% CI-DESSOUS: conversion des caractères accentués UTF-8 
% en caractères TeX dans les listings...
\lstset{
  literate=%
  {À}{{\`A}}1 {Â}{{\^A}}1 {Ç}{{\c{C}}}1%
  {à}{{\`a}}1 {â}{{\^a}}1 {ç}{{\c{c}}}1%
  {É}{{\'E}}1 {È}{{\`E}}1 {Ê}{{\^E}}1 {Ë}{{\"E}}1% 
  {é}{{\'e}}1 {è}{{\`e}}1 {ê}{{\^e}}1 {ë}{{\"e}}1%
  {Ï}{{\"I}}1 {Î}{{\^I}}1 {Ô}{{\^O}}1%
  {ï}{{\"i}}1 {î}{{\^i}}1 {ô}{{\^o}}1%
  {Ù}{{\`U}}1 {Û}{{\^U}}1 {Ü}{{\"U}}1%
  {ù}{{\`u}}1 {û}{{\^u}}1 {ü}{{\"u}}1%
}

%%%%%%%%%%
% TAILLE DES PAGES (A4 serré)

\setlength{\parindent}{0pt}
\setlength{\parskip}{1ex}
\setlength{\textwidth}{17cm}
\setlength{\textheight}{23cm}
\setlength{\oddsidemargin}{-.7cm}
\setlength{\evensidemargin}{-.7cm}
\setlength{\topmargin}{-.5in}

%%%%%%%%%%
% EN-TÊTES ET PIED DE PAGES

\pagestyle{fancyplain}
\renewcommand{\headrulewidth}{0pt}
\addtolength{\headheight}{1.6pt}
\addtolength{\headheight}{2.6pt}
\lfoot{}
\cfoot{}
\rfoot{\footnotesize\sf page~\thepage/\pageref{LastPage}}
\lhead{\footnotesize\sf Projet de spécialité}
\rhead{\footnotesize\sf ENSIMAG-2A} % numéro d'équipe Teide 

% titre, auteur et date
\title{Rapport du projet de spécialité\\
  Passerelle Git-MediaWiki, environement de test\\
  ~\\
  \includegraphics[scale=0.75]{logo_ensimag.jpg} 
}

%% penser à indiquer: numéro d'équipe Teide + noms + groupe TD
\author{Simon Cathébras, Julien Khayat, Simon Perrat, Charles Roussel,
  Guillaume Sasdy}
\date{Le \today}
\begin{document}
\maketitle

\section{Travail réalisé}

Le travail réalisé durant ce projet est un environnement de test pour
la passerelle \textit{Git-MediaWiki}, un projet initié l'année
dernière à l'Ensimag.

\subsection{La passerelle  \textit{Git-MediaWiki}}

Cet outil a pour but de permettre de gérer un MediaWiki a l'aide de
Git. Pour bien comprendre de quoi il retourne, nous allons expliquer
précisément ce que sont Git et MediaWiki

\subsubsection{MediaWiki : des sites communautaires à contribution libre}

Le terme wiki désigne un site web dont les pages peuvent être éditées
par les internautes qui le fréquentent, et MediaWiki est un type de
wiki particulier. Wikipédia, le wiki le plus fréquenté actuellement,
fonctionne avec MediaWiki, et on y retrouve toutes ses
caractéristiques, notamment~:


\begin{itemize}
\item chacun peut y contribuer de manière libre~;
\item possibilité d'ajouter, supprimer, et surtout éditer des pages~;
\item conservation de l'historique complet des modifications.
\end{itemize}

Le principal défaut de ces sites est que leur édition peut s'avérer
fastidieuse, et plusieurs problèmes se présentent régulièrement aux
gros utilisateurs~:

\begin{itemize}
\item conflits entre éditions concurrentes~;
\item la moindre petite modification de page, comme une correction
  orthographique, engendre une entrée dans l'historique~;
\item une seule révision de page entraîne une révision complète du
  wiki, l'historique est donc rapidement surchargé~;
\item ainsi pour une mise à jour importante (par exemple sur Wikipédia, un
  évènement d'actualité marquant), les pages touchées doivent être
  révisées une par une~;
\item de plus, dans le cas d'une modification conséquente sur une
  page, on est confronté à un dilemme~:
  \begin{itemize}
  \item soit on enregistre régulière et on produit un historique "sale"
  \item soit on n'enregistre qu'une fois le travail terminé, et on
    s'expose alors au risque de perdre son travail sur une coupure de
    connexion, un crash, une fermeture accidentelle du navigateur...
  \end{itemize}
\end{itemize}

Le but de la passerelle est alors de proposer une solution à ces
problèmes en permettant d'utiliser le gestionnaire de version
\textit{Git} pour travailler sur un MediaWiki.

\subsubsection{Git : un gestionnaire de version}

Un gestionnaire de version est un logiciel qui permet de partager et
synchroniser un ensemble de fichiers entre plusieurs
utilisateurs. Typiquement, un dépot est créé sur un serveur et toute
personne qui y a accès peut le copier sur son ordinateur personnel,
pour ensuite partager ses modifications sur les
fichiers. \textit{Git}, comme d'autres gestionnaires de version,
propose en outre les fonctionnalités suivantes~:

\begin{itemize}
\item possibilité d'avoir plusieurs versions d'un même fichier~;
\item gestion des éditions concurrentes~;
\item conservation d'un historique complet et détaillé.
\end{itemize}

On constate alors quelques avantages par rapport à un wiki~:

\begin{itemize}
\item on peut éditer un fichier en plusieurs fois en faisant une seule
  mise en ligne, l'historique résultant est alors plus propre~;
\item faire une mise à jour simultanée de plusieurs fichiers est plus simple~;
\item aucun risque de perte de donnée en cours d'édition, car la plupart
  des éditeurs de texte ont un système de sauvegarde automatique.
\end{itemize}

\subsection{L'environnement de tests}

L'outil \textit{Git-MediaWiki} est actuellement intégré à \textit{Git}
comme une contribution, sans être présent dans son coeur de
commandes. Pour pouvoir aller plus loin et intégrer nativement la
passerelle dans \textit{Git}, elle doit être munie d'une base de test
conséquente pour valider son fonctionnement. Jusqu'à présent, les
tests devaient être effectués à la main.

L'objectif de notre groupe est de construire un environnement complet
de tests pour la passerelle, avec pour objectif, à court terme de
valider les fonctionnalités existantes et d'en développer de
nouvelles, à long terme de pousser \textit{Git-MediaWiki} jusque
dans coeur de \textit{Git}.

\section{La communautée \textit{Git}}

\textit{Git} est un logiciel libre et open source, ce qui signifie que
tout le monde a accès à son code source et peut y contribuer en
proposant des modifications. Il existe alors une communauté de
développeurs contribuant à \textit{Git}, qui communiquent ensemble via
une mailing-list dédiée et échangent ainsi sur les bugs rencontrés et
leur correction, les ajouts de fonctionnalités et les mises à jour du
code de manière générale. La version officielle est maintenue par
Junio~C.~Hamano, qui se charge de vérifier que les patches proposés
ont un niveau de finition suffisant pour être intégrés, en plus d'être
lui-même un des contributeurs principaux au code source de \textit{Git}.

\subsection{Organisation}

Le code de \textit{Git} est disponible librement sur une archive
\textit{Git}. Cette archive comprend trois branches, qui correspondent
chacune à différents avancements des patches proposés. La principale,
\textit{master}, est celle qui correspond à la version officielle
distribuée. La branche \textit{next} contient les patches qui seront
ajoutés à la prochaine version officielle. Enfin, la branche
\textit{pu}, pour \textit{proposed updates}, comprend les patches qui
sont en voie d'intégration à \textit{Git}, mais qui ont encore besoin
de peaufinage pour passer dans \textit{next}. Il faut noter qu'avant
même qu'un patch puisse passer dans cette branche, il doit déjà être
relu, testé et approuvé par la communauté sur la mailing-list.

\subsection{Interactions avec la communautée}

Nos échanges avec la communauté se font via la mailing-list de
\textit{Git}. Nos propositions de patches se font en plusieurs temps~:
tout d'abord, nous envoyons sur cette liste de diffusion une `cover
letter`, soit un mail où nous présentons notre équipe et notre
projet. Lorsque nos premiers travaux nous semblent assez mûrs, nous
envoyons alors un série de patches sur la mailing list, que la
communauté peut alors relire et commenter. Nous tenons alors compte
des critiques émises pour corriger notre code, et envoyons une
nouvelle série de patches. Nous continuons ainsi jusqu'à obtenir un
code qui atteigne les critères d'acceptation par \textit{Git}, et qui
soit en mesure d'être ajouté à la branche \textit{pu}.
Actuellement, nous n'avons pas atteint cette branche mais avons bon
espoir d'y parvenir.

\section{Difficultées pendant le projet}

\subsection{Premier Run : Lundi 21 Mai $\rightarrow$ Mardi 29 Mai}

Le premier run de ce projet de spécialité a comporté un nombre conséquent de problèmes, tant au niveau organisationnel que technique

\subsubsection{problèmes techniques}

Durant le premier run, nous avons eu des difficultés à appréhender le projet. Nous avons dû procéder à beaucoup de changements de structure pour notre code
\begin{enumerate}
\item plusieurs fichiers \textit{bash} encapsulant du \textit{Perl}
\item continuer les étapes
\end{enumerate}

\subsubsection{problèmes organisationnels}

Après une analyse à la fin du premier run, nous avons déterminé plusieurs problèmes d'organisation : 

\begin{itemize}
\item manque de communication dans le groupe
\item mauvaise compréhension de certains points de la charte d'équipe par une partie du groupe.
\item plusieurs membres du groupe ont du s'absenter un après midi ou une journée pour des raisons personnelles
\item pas d'évènement de cohésion de groupe
\end{itemize}

Tout ces points conduisant à un premiers run assez désorganisé, produisant des difficultés à apréhender le projet sur un plan général.

\subsection{Deuxième Run : Mardi 29 Mai $\rightarrow$ Mardi 5 Juin}

Pour le second run, on rencontre beaucoup moins de difficultées. Quelques problèmes techniques ont parfois ralentit la progression du projet, mais jamais de manière drastique. Notre plus gros problème a été d'avoir sous-estimé le temps que prendrais d'apporter les correctifs demandés par la mailling list. 

Le travail prévu à cette fois été effectué à temps et correctement.

\subsection{Troisième Run : Mardi 5 juin $\rightarrow$ Mercredi 13 juin}

\section{Bilan}

\subsection*{Au niveau de l'équipe}

\subsection*{Au niveau des enseignants}

L'ensemble de l'équipe s'accorde à dire que l'enseignant de référence ( Matthieu Moy ) a été d'une qualité exemplaire tout au long du projet. Disponible, réactif sur les questions par mail et présent dans la salle. Il aura été d'une grande aide grace à sa connaissance des l'environnement du projet, c'est à dire la communauté Git et ses coutumes, le logiciel libre dans un sens plus général, mais également dans sa capacité à nous orienter dans la bonne direction en ce qui concerne l'architecture des fichiers. 

Un autre point très positif du projet aura été l'application des méthodes agiles, qui ont permit à l'équipe de travailler plus sereinement. Les passages régulier de XXXXXX nous ont incité à réfléchir à notre organisation, augmentant notre efficacité au cours du projet. 

\subsection*{Améliorations}

Après discussion dans l'équipe voici quelques pistes d'amélioration pour le projet dans les années à venir : 

\begin{itemize}
\item donner des informations ``technique'' sur le projet sur la page de présentation. Il aurait put être interessant de connaitre les langages utilisés à l'avance. 
\item 
\end{itemize}

\end{document}
